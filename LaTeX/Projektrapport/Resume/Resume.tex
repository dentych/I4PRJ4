\section*{Resumé}
Dette projekt er udviklet af seks studerende fra Ingeniørhøjskolen, Aarhus Universitet. Formålet er at lave et tværfagligt projekt, der dækker de fem udbudte kurser på IKT-studiets 4. semester. Det har samtidigt haft til formål at skabe praktisk erfaring med agil softwareudvikling, og dertilhørende test, versionsstyring mm.\\

I projektet er der udviklet en funktionel prototype af et digitalt salgssystem, hvor et eller flere kasseapparater kan forbindes. Derudover er der udviklet et administrationssystem, hvor superbrugere kan vedligeholde produktkateloget, samt en central server, hvor kasseapparater og administrationssystemer kan forbinde til. Kasseapparat og administrationssystem er udviklet med grafiske brugergrænseflader.\\

Igennem processen er der anvendt en iterativ udviklingsprocess i form af en afart af SCRUM for at holde projektet agilt. Dette har sikret muligheden for at udvikle på forskellige delsystemer sideløbende. Produktet er blevet udviklet i fem iterationer over 14 uger. Der er anvendt Jenkins til continuous integration samt til at holde styr på CodeCoverage. Der er benyttet Git til versionsstyring af kildekode, dokumentation og rapport.


\section*{Abstract}
This projet has been developed by six students from Aarhus School of Engineering (Aarhus University). The purpose is to create a interdisciplinary projekt that covers the five courses that has been offered at the 4th semester on the ICT-studies. More than that, this project has the purpose of giving practical experience within agile softwaredevelopment, including but not limited to tests, version control and so forth.\\
Through the development a variant of SCRUM has been used to keep the project agile, contemprary jenkins has been used for continous integration and git for version control.\\
A functional prototype of a point-of-sale system has been developed which comply with with projects demands and specifications. This point-of-sale system has been developed over 5 itterations through 14 weeks. The point-of-sale system has been developed using 3 individuel systems, Checkout counter, centralserver and administrationssystem and a shared package containing common logic.

Conclusionshizzle?