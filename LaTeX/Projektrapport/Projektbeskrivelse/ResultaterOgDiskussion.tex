\section{Resultater og Diskussion}
Projektet har resulteret i en funktionel prototype som overholder alle "skal" punkter fra kravspecifikationen der blev opstillet til systemet på forhånd.\\

\textbf{Kasseapparat}\\
Kasseapparatet kom til at opfylde de Use Cases som dette var en del af. Det kan gennemføre salg og danne salgskvitteringer for disse. Det kan som en del af et salg tage vare retur. Kasseapparatet kan kommunikere med CentralServer, hvorfra det kan modtage en oversigt over de produkter som kan sælges, samt sender en oversigt for hvert salg der er foretaget.\\
Der er desuden lagt tid i det grafiske ved kasseapparatet, med henblik på hurtig betjening, ved adskillese af de forskellige segmenter og knapper.\\

\textbf{Administrationssystem}\\
Administrationssystem var en del af de fleste af de opstillede Use Cases, og havde derfor en masse logik der skulle laves. Oprettelse, sletning og redigering af produkter og kategorier var en del af minimumssystemet, og al denne funktionalitet er funktionsdygtigt i prototypen.\\
Der er brugt meget tid på at udvikle et produkt der nemt kan skaleres med flere instanser af Administrationssystem og Kasseapparat. Der blev således brugt en del tid på at udvikle funktionalitet til at holde alle instanser opdateret hele tiden, ved hjælp af asynkron kommunikation.\\
I Administrationssystem er der hovedsagelig lagt vægt på funktionalitet, hvilket betyder at det visuelle design har været sekundær prioritet. Administrationssystem gør i prototypen blot brug af det standard design der ligger i Windows.\\

\textbf{CentralServer}\\
CentralServer kan kommunikere med både Administrationssystem og Kasseapparat. CentralServer har også funktionalitet til at oprette, redigere og slette produkter og kategorier på kommando fra Administrationssystem.\\

\textbf{SharedLib}\\
Alle tre systemer gør brug af et fælles bibliotek, kaldet SharedLib, hvor funktionalitet til at encode og decode kommandoer til XML ligger. Dette samler funktionaliteten, så delsystemerne ikke selv skal implementere den samme funktionalitet flere gange.\\
Når al funktionaliteten er samlet et sted betyder også, at alle tre delsystemer kommunikerer på præcis samme måde, så der altid er vished om, at protokollen mellem delsystemerne overholdes.\\