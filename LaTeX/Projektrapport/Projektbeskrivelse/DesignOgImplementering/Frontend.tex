\subsection{Kasseapparat}

Kasseapparatet blev bygget ved brug af 3 layers. Presentation-, Business logic- og Data Access Layer. Disse vil blive beskrevet herunder.

\subsection{Designovervejelser}
Designet af Kasseapparatet startede ud med at der blev tegnet en skitse af hvordan selve brugergrænsefladen skulle blive opbygget.
Det var vigtigt at der en klar opdeling af de forskellige lag. Dette betød at der mellem præsentations- og business logic laget blev benyttet MVVM. MVVM er et pattern, der sikrer at kommunikationen mellem grænsefladen er håndteret ordentligt. \\
Det første der blev implementeret til Kasseapparatet er grænsefladen samt de underliggende viewmodels. Disse havde en klar forbindelse til de elements de ligger i grænsefladen igennem commands. Og da der ikke var tilknyttet funktionalitet kunne disse klasser opsættes uden yderligere design, der kom dog Viewmodels senere, der stod for at styre knapperne og disse blev der selvfølgelig lavet sekvensdiagrammer for.l

\subsubsection{Presentation Layer}

Det første der blev opsat til kasseapparatet var grænsefladen. Dette kan virke lidt bagvendt, men det var fra starten et fokus at skabe et anderledes kasseapparat der var nemt at benytte sig af. Med dette menes der et program med en klar vej fra start til slut. Der måtte gerne være flere måder at gennemføre salg på, men brugeren måtte ikke være tvunget til at benytte sig af allesammen.
Først blev der opsat en skitse, se figur, hvor kasseapparatet var delt op i 3 segmenter. Her var idéen at produkterne skulle være i segmentet på venstre side, indkøbskurven i midten, og knapperne til brug i betalingen på højre side. Derved ville et salg gå fra højre i mod venstre i det at man først ville tilføje et produkter i venstre side. Disse ville komme frem i midten, og til sidst ville betalingen blive gennemført i højre side af kasseapparatet. 

\begin{figure}[H]
	\centering
	\includegraphics[width=0.9\linewidth]{Projektbeskrivelse/DesignOgImplementering/pics/GUI}
	\caption{Endelig Implementering}
	\label{fig:sub2}
\end{figure}

Som man kan se på figur \ref*{fig:sub2} så blev grænsefladen udviklet relativt tæt på skitsen, dog med nogle flere knapper i betalingssegmentet. For at sikre at der var mange måder at tilføje produkter på, blev det lavet sådan at man kunne tilføje et produkt ved at klikke gentagne gange på hver produkt. Samtidig kan man trykke et et tilføjet produkt, skrive et antal i betalingsdelen og trykke på mængde. Så vil den angivne mængde af produkter blive tilføjet. \\

\textbf{Knapper og sideskift} \\
Noget der er lagt meget tid i, er knappeskift. Dette blev implementeret ved at oprette et kontrolobjekt til knapper\footnote{ProductButtonControl}. Dette objekt indeholder en liste der inderholder liste af 12 knapper. Derved kan der nemt skiftes side af knapper bare ved at ændre den liste der vises på kasseapparatet.
\subsubsection{Business Logic Layer}
Kasseapparatets Business Logic Layer, består af 2 overordnede dele. ProductCategoryList og ShoppingList. \\

ProductCategoryList er implementeret med henblik på at indeholde produkt kategorier, hentet fra CentralServer, og dermed produkterne. ProductCategoryList er implementeret som ObservableCollection~\cite{ObsCol}, så de ViewModels som benytter denne, kan håndtere den som en liste.\\

ShoppingList er implementeret som en ViewModel fra MVVM. Modellen til ShoppingList er PurchasedProduct som findes i SharedLib. Viewet som præsenterer ShoppingList er MainWindow. \\
ShoppingList indkorporerer ObservableCollection~\cite{ObsCol}, for at ShoppingList kan håndteres som en liste af PurchasedProducts. \\
ShoppingList har desuden en række commands\footnote{RelayCommands~\cite{RelayC}}, som bindes til midtersegmentets knapper samt annuller knappen i det højre segment. Disse commands er også valgt for at der fra ShoppingList kan kontrolleres hvornår disse knapper er enabled/disabled.\\
Til ShoppingList høre der også view code som kalder funktioner i ShoppingList med information fra displayet, dette være sig vare antal eller betalingssum.\\
\subsubsection{Data Access Layer}
Data Access Layer består af tre dele. Den første består af at modtage produktoversigter fra CentralServer. Den anden består af at sende information om gennemførte køb til CentralServer. Den tredje består af at denne salgskvitteringer efter køb er blevet gennemført

\begin{figure}[H]
	\centering
	\includegraphics[width=\textwidth]{Projektbeskrivelse/DesignOgImplementering/Frontend/Pics/DALsq}
	\caption{Kommunikation til/fra CentralServer}
	\label{fig:KAtCS}
\end{figure}

Figur \ref{fig:KAtCS} viser et generelt eksempel på hvordan kommunikationen til CentralServer fungere. Ekspedienten gennemføre f.eks. et køb i brugergrænsefladen, herefter laves et kommando objekt, som encodes til en XML-streng ved hjælp af SharedLib. Denne XML streng sendes til CentralServer. Forventes et svar modtages dette i form af en XML string, denne decodes til kommando objekt, som inderholder informationen.\\
Oprettelse af salgskvitteringer foregår ved at der oprettes en tekst fil, navngivet efter tidspunkt for købet.




