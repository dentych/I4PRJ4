\chapter{Indledning}
Dette 4. semester projekt omhandler et digitalt salgssystem. Emnet er valgt, da skolens fredagsbar manglede et kasseapparat, og projektet blev her tilgængeligt for 4. semester. Gruppen mente at det kunne være spændende at arbejde med et projekt, hvor der var en reel kunde tilknyttet. Gruppen blev imidlertid enige om at gå sin egen vej, da det gav frihed til at kunne indkorporere teori fra samtlige 4. semester fag og samtidigt sikrede at gruppen kunne udvikle et produkt, der var direkte udledt af egne idéer. \\

Det var et krav til projektet, at gruppen benyttede en iterativ process. Dette harmonerede godt med, at gruppen ønskede at ende ud med et fungerende produkt og derfor blev en beslutning taget om at første iteration skulle ende ud i et fungerende minimumssystem. \\

Til at forklare systemet er der udbygget en projektdokumentation, hvor der bruges 4+1 sammen med UML til at beskrive de grundlæggende elementer i systemet. Hvordan softwaren er designet og implementeret kan ligeledes læses i projektdokumentationen.
Denne rapport vil derfor ikke give en fuldt uddybende beskrivelse af selve opbygningen af systemet, men fortælle om de tanker gruppen har gjort sig igennem forløbet.