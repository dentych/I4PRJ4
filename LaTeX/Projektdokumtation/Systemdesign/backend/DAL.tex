\subsection{Data Access Layer}


\subsection{Data Access Layer} \label{DAL}

\subsection{Introduktion}


Administrationssystemet er en applikation som kan tilføje, fjerne og redigere produkter eller kategorier, tilhørende kassesystemets database.\\
Applikationen er designet skalerbart, således brugeren ikke er begrænset til kun at kunne tilføje, fjerne eller redigere fra udelukkende en node, men derimod synkront holde databasen opdateret fra flere noder på en gang. Dette betyder at databasen og alle opkoblede administrationssystemer, altid vil være helt synkroniseret.

\subsubsection*{Design og struktur}
Backend er designet på baggrund af MVVM designmønstret~\cite{MVVM}. 
\begin{figure}[!h]
    \centering
    \includegraphics[width=0.8\textwidth]{Systemdesign/backend/Images/Opbygning.png}
    \caption{Oversigt over alle namespaces i administrationssystemet}
    \label{fig:oversigtAs}
\end{figure}

På figur \ref{fig:oversigtAs} ses det hvordan strukturen i administrationssystemet er opbygget. 

\begin{itemize}
\item \textbf{Views} indeholder XAML og codebehindfilerne for den grafiske brugerflade. 
\item \textbf{ViewModels} indeholder de data og commands som views binder til. 
\item \textbf{Models} indeholder datamodeller og businesslogik.
\item \textbf{Datamodels} indeholder datamodeller, herunder bl.a. produkter, collections og kategorier.
\item \textbf{Events} indeholder events der bruges til kommunikation mellem viewmodel, den tilhørende aggregator samt parametremodellen til nogen af disse events.
\item \textbf{SocketEvents} indeholder de events der bliver raised af Client ved modtagelse af data.
\item \textbf{Brains} indeholder generel businesslogik (???) til bl.a. at oprette kommandoer til klienten, og fejl-håndtering.
\item \textbf{Dependencies} indeholder sourcekode til at arbejde med commands.
\item \textbf{Communication} indeholder klient til at skabe forbindelse til Central Server.
\end{itemize}

De ovenstående 

\subsubsection{Klassebeskrivelser}
\textbf{FilePrinter}

FilePrinters ansvar er at danne en salgskvitering efter et salg er gennemført. Salgskvitteringen bliver dannet som en simpel tekst fil, navngivet efter dato og klokkeslet for købet. Denne tekst fil vil så i fremtidig arbejde kunne udskrives som en kvitering til kunden. 

\bigskip
\textbf{Connection}

Connection kan oprette og nedlægge en TcpClient, som forbinder \gls{KA} op til \gls{CS}. Desuden kan Connection sende en tekst string over den oprettede TcpClient, samt læse en tekst string her fra.

\begin{figure}[H]
    \centering
    \includegraphics[]{Systemdesign/Frontend/DAL/Pics/Connection}
    \caption{Connection}
    \label{fig:Connection}
\end{figure}

\bigskip
\textbf{DBcontrol}

DBcontrols ansvar er at sende korrekt formatterede string til Connection. DBcontrol benytter \gls{SL} til at oprette kommando objecter med den infomation som skal sendes, samt at encode/decode disse til XML strings.

\bigskip
\textbf{FakeDBcontrol}

FakeDBcontrol er opretter med henblik på at kunne test \gls{KA}, uden at der er forbindelse til \gls{CS}. FakeDBcontrol giver adgang til en række test produkter.

\begin{figure}[H]
    \centering
    \includegraphics[]{Systemdesign/Frontend/DAL/Pics/FakeDBcontrol}
    \caption{FakeDBcontrol}
    \label{fig:FakeDBcontrol}
\end{figure}

\bigskip

\subsubsection{Kommunikation med Central Server}
Klienten i Administrationssystemet består af en socketconnection som er defineret i SharedLib[Ref]. Da denne skal bruges i både business logic, men også i MainWindowViewModel og der samtidig ønskes at den samme benyttes hver gang, bruges Singleton~\cite{SINGLETON} for denne forbindelse. \\
Til at sende data benyttes en ModelHandler, som hver i sær har en metode til de forskellige handlinger. Eksempelvis EditProduct, DeleteProduct mf. Disse metoder arbejder alle sammen på samme måde:
\begin{enumerate}
\item Opret en XML-kommandostring vha. protokollen\footnote{Nærmere beskrivet under SharedLib, afsnit \ref{SHAREDLIB}, side \pageref{SHAREDLIB}}
\item Send data til klienten
\end{enumerate}


\begin{figure}[!h]
    \centering
    \includegraphics[width=1\textwidth]{Systemdesign/backend/Images/DataSend.png}
    \caption{Eksempel på hvordan CreateProduct bliver behandlet i systemet og sendt til Central Server}
    \label{fig:CreateSend}
\end{figure}

På figur \ref{fig:CreateSend} ses det hvordan en viewmodel kunne kalde CreateProduct i ModelHandler\footnote{Nærmere beskrevet under klassebeskrivelser, afsnit \ref{Modelhandler_Beskrivelse} side \pageref{Modelhandler_Beskrivelse}}, som får lavet en XML-streng \gls{CS} kan forstå. Derefter bliver denne data sendt ud igennem en socket forbindelse.
Derefter afsluttes der. Det betyder at når der eksempelvis oprettes et nyt produkt, bliver der ikke opdateret noget i systemet, der bliver udelukkende sendt data til \gls{CS}.\\\\


For at modtage data, subscriber MainWindowViewModel på nogle forskellige events, som eksempelvis OnProductCategoryDeleted, OnProductEdited mf. Dette gøres igennem klassen SocketEventHandlers\footnote{Nærmere beskrevet under klassebeskrivelser, afsnit \ref{SocketEventHandlerBeskrivelse} side \pageref{SocketEventHandlerBeskrivelse}}, som også indeholder de eventhandlere der bliver kaldt, når et event raises på serveren.  Det er derfor disse eventhandlere som håndtere dataen, og lægger den nye data ind i datamodellerne – og det er først der, der bliver opdateret lokalt. På den måde vil der aldrig eksistere noget i databasen der ikke eksisterer i Administrationssystemet og vice versa. 
\begin{figure}[!h]
    \centering
    \includegraphics[width=1\textwidth]{Systemdesign/backend/Images/DataReceive.png}
    \caption{Eksempel på hvordan produktet igen bliver modtaget fra serveren, efter det er blevet oprettet.}
    \label{fig:DataReceive}
\end{figure}    

På figur \ref{fig:DataReceive} ses det hvordan et event bliver raised, og den rigtige eventhandler bliver kaldt, på baggrund af de subscribtions der blev lavet i MainViewViewModel. Først efter at denne har tilføjet produktet til den interne produktliste, er produktet synlig i programmet. Det er derfor \gls{CS} der bestemmer hvornår et produkt blive tilføjet, fjernet eller redigeret.\\
\\
Dette er gjort således at der ikke, ved en fejl, kunne opstå problemer med at et produkt i \gls{AS} ikke stemmer overens - eller overhovedet eksisteret - i \gls{CS}. Desuden tillader dette også at flere noder kan lytte på disse events, så hvis en anden node eksempelvis tilføjer et produkt, så bliver listen i alle andre noder også opdateret. Det er besluttet at der skal subscribes på events enkeltvis, således det er muligt at have noder der kun er interesseret i en del af dataen, kun at få denne data. 
Der kan derfor sikres at \textit{alle} med forbindelse til \gls{CS} er fuldt opdateret på de ting de subscriber på.

