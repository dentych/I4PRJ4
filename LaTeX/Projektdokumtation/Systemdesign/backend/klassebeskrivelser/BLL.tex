\subsubsection{Klassebeskrivelser}

Herunder ses en  beskrivelse af samtlige klasser i business logic layer. Beskrivelsen vil fremgå som \textit{klassenavn"} efterfulgt af en kort beskrivelse af klassens ansvar og funktionalitet.\\\\


\textbf{Brains} namespacet indeholder generel businesslogik for at kunne udføre handlinger, så som at oprette kommandoer, parse objekter til XMl mv. Den benytter sig meget af det delte bibliotek \gls{SL}.\\

\paragraph{ViewModels} namespacet indeholder viewmodels til alle views. Viewmodels indeholder den data som præsentationslaget binder til. Dette indebærer kommandoer til knapper, samt properties som skal bruges af viewet.

\textbf{MainWindowViewModel}\\
Denne viewmodel hører til MainWindow, som er hovedvinduet i \gls{AS}. Denne viewmodel indeholder alle produktkategorier, som har lister med produkter. Derudover er der kommandoer til at åbne alle andre vinduer i systemet. Denne viewmodel har subscribed på alle andre viewmodels "Loaded" events. Se nærmere omkring disse under \ref{eventaggregator_beskrivelse}.
\begin{figure}[H]
	\centering
	\includegraphics[width=0.2\textwidth]{Systemdesign/backend/klassebeskrivelser/Images/Error.png}
	\caption{Error}
	\label{fig:modelhandler}
\end{figure}

\paragraph{Brains} namespacet indeholder generel businesslogik for at kunne udføre handlinger, så som at oprette kommandoer, parse objekter til XMl mv. Den benytter sig meget af det delte bibliotek \gls{SL}.\\

\textbf{Error}\\
Errorklassen er en fejlbeskedprinter, som kan printe en besked på skærmen, eksempelvis ved fejl eller notifikationer.
\begin{figure}[!h]
    \centering
    \includegraphics[width=0.2\textwidth]{Systemdesign/backend/klassebeskrivelser/Images/Error.png}
    \caption{Error}
    \label{fig:error}
\end{figure}
 \bigskip 
 
 
 
 
 
 
 


\textbf{ModelHandler}\\
ModelHandlerens ansvar er at lave den kommando der skal sendes til \gls{CS}. Denne kommando kan være til for eksempelvis at oprette eller nedlægge et produkt. Denne klasse er desuden den klasse, som kalder klienten til at sende den netop oprettede kommando. 
\begin{center}
\begin{figure}[!h]
    \centering
    \includegraphics[width=0.45\textwidth]{Systemdesign/backend/klassebeskrivelser/Images/ModelHandler.png}
    \caption{ModelHandler}
    \label{fig:modelhandlerreal}
\end{figure}
\end{center}
\label{Modelhandler_Beskrivelse}
 \bigskip
 
 
 
 
 

\textbf{PrjProtokol}\\
PrjProtokol har til ansvar at lave kald til \gls{SL} og dens protokol.Den sørger udelukkende for at få lavet de rigtige kald, for at få et eventuelt objekt parset om til en XML-streng, igen. \bigskip
\begin{center}
\begin{figure}[!h]
    \centering
    \includegraphics[width=0.45\textwidth]{Systemdesign/backend/klassebeskrivelser/Images/PrjProtokol.png}
    \caption{ModelHandler}
    \label{fig:prjprotko}
\end{figure}
\end{center}
\label{PrjProtokol_Beskrivelse}
 \bigskip
 




\bigskip
\bigskip




\textbf{Events} namespacet indeholder klasser, som fungerer som events, til brug i kommunikation mellem viewmodels. Der vil i dette afsnit beskrives flere klasser i en beskrivelse, da de er så ens som de er. Der er desuden dataobjekter i dette namespace, til brug i sammenhæng med events.\\
\bigskip

\textbf{EditProductParameters, EditCategoryParms, DeleteCategoryParms}\\
Et objekt indeholdende data der bliver brugt i en anden viewmodel, når NewEditProductData eventet raises. Dette bruges da kun én parameter kan bruges i disse events. Kommunikation mellem viewmodels er beskrevet nærmere på side \pageref{viewcomm}. \bigskip
\begin{center}
\begin{figure}[!h]
    \centering
    \includegraphics[width=0.50\textwidth]{Systemdesign/backend/klassebeskrivelser/Images/Parms.png}
    \caption{Paramtereklasser}
    \label{fig:EditProductParameters}
\end{figure}
\end{center}
\label{EditProductParameters_Beskrivelse}
 \bigskip 




\textbf{CategoryListUpdated, NewEditProductData, NewEditCategoryData, NewDeleteCategoryData}
Disse arver ale sammen fra PubSubEvent\footnote{En eventtype som bruges i PRISM ~\cite{PRISM}}. Disse events raises når et vindue med en anden viewmodel er loaded, og der ønskes at dele data mellem disse viewmodels. Kommunikation mellem viewmodels er beskrevet nærmere på side \pageref{viewcomm}.  \bigskip
\begin{center}
\begin{figure}[!h]
    \centering
    \includegraphics[width=0.50\textwidth]{Systemdesign/backend/klassebeskrivelser/Images/Events1.png}
    \caption{Events}
    \label{fig:CategoryListUpdated}
\end{figure}
\end{center}
\label{CategoryListUpdated_Beskrivelse}
 \bigskip 




\textbf{AddProductWindowLoaded, DeleteCategoryWindowLoaded, EditCategoryWindowLoaded, AddCategoryWindowLoaded, EditProductWindowLoaded}
Disse klasser arver ale sammen fra PubSubEvent\footnote{En eventtype som bruges i PRISM ~\cite{PRISM}}. De fungerer således som et event de raises, når et vindue loades, således at der kan deles data. Kommunikation mellem viewmodels er beskrevet nærmere på side \pageref{viewcomm}. \bigskip
\begin{center}
\begin{figure}[!h]
    \centering
    \includegraphics[width=0.50\textwidth]{Systemdesign/backend/klassebeskrivelser/Images/Events2.png}
    \caption{Events}
    \label{fig:AddProductWindowLoaded}
\end{figure}
\end{center}
\label{AddProductWindowLoaded_Beskrivelse}
 \bigskip 


\bigskip
\bigskip

\textbf{SocketEvents} namespacet indeholder klasser til brug i forbindelse med events raised fra \gls{SL}.\\
\bigskip

\textbf{SocketEventHandlers}\\
Denne klasse indeholder alle eventhandlere som bruges, når der bliver raised et event fra \gls{SL}'s sockets. Dette kunne eksempelvis være et event der fortæller at produktlisten er opdateret eller lignende. Den har desuden subscribemetoder, som bruges til at subscribe på disse events.
\begin{center}
\begin{figure}[!h]
    \centering
    \includegraphics[width=0.50\textwidth]{Systemdesign/backend/klassebeskrivelser/Images/SocketEvents.png}
    \caption{SocketEventHandlers}
    \label{fig:SocketEventHandlers}
\end{figure}
\end{center}
\label{SocketEventHandlerBeskrivelse}
 \bigskip 
 
 
\textbf{SingleEventAggregator}\\
Denne klasse indeholder en EventAggregator fra PRISM~\cite{PRISM}, som man håndterer alle events, publishers og subscribers. Denne er implementeret som en singleton, således alle der bruger denne, bruger den samme.
\begin{center}
\begin{figure}[!h]
    \centering
    \includegraphics[width=0.30\textwidth]{Systemdesign/backend/klassebeskrivelser/Images/agg.png}
    \caption{SingleEventAggregator}
    \label{fig:SingleEventAggregator}
\end{figure}
\end{center}
\label{SingleEventAggregator_Beskrivelse}
 \bigskip 

\bigskip
\bigskip

\textbf{Datamodels} namespacet indeholder alle datamodeller. Det er disse datamodeller hele systemet benytter.\\
\bigskip

\textbf{BackendProductCategoryList}\\
Denne klasse indeholder en liste af kategorier. Det er denne datatype som indeholder al data fra \gls{DB} , når det hentes ned fra \gls{CS}. Denne klasse arver fra en anden klasse, ASyncObservableCollection, som er skrevet af Thomas Levesque~\cite{ASYNC}. Dette har været nødvendigt, da der i nogle tilfælde ville være flere tråde der arbejdede på denne collection.
\begin{center}
\begin{figure}[!h]
    \centering
    \includegraphics[width=0.30\textwidth]{Systemdesign/backend/klassebeskrivelser/Images/BPCList.png}
    \caption{BackendProductCategoryList}
    \label{fig:BackendProductCategoryList}
\end{figure}
\end{center}
\label{BackendProductCategoryList_Beskrivelse}
 \bigskip 
 
 \textbf{BackendProduct}\\
Denne klasse arver fra \gls{PD} fra \gls{SL}. Klassen implementerer desuden fra INotifyProperyChanged, og derved kan dynamisk opdatere i den grafiske brugerflade. Denne model symboliserer \gls{PD}.
\begin{center}
\begin{figure}[!h]
    \centering
    \includegraphics[width=0.30\textwidth]{Systemdesign/backend/klassebeskrivelser/Images/Backendproduct.png}
    \caption{BackendProduct}
    \label{fig:BackendProduct}
\end{figure}
\end{center}
\label{BackendProduct_Beskrivelse}
 \bigskip 
 
 
  \textbf{ConnectionString}\\
Denne klasse bruges således at der kan fremvises i GUI'en, om der er forbundet til \gls{CS}. Denne implementerer INotifyProperyChanged, således visningen kan opdateres dynamisk.
\begin{center}
\begin{figure}[!h]
    \centering
    \includegraphics[width=0.30\textwidth]{Systemdesign/backend/klassebeskrivelser/Images/const.png}
    \caption{ConnectionString}
    \label{fig:ConnectionString}
\end{figure}
\end{center}
\label{ConnectionString_Beskrivelse}
 \bigskip 


 \textbf{BackendProductCategory}\\
Denne klasse arver fra \gls{PD} fra \gls{SL}. Klassen implementerer desuden fra INotifyProperyChanged, og derved kan dynamisk opdatere i den grafiske brugerflade. Denne model symboliserer \gls{PDK}.
\begin{center}
\begin{figure}[!h]
    \centering
    \includegraphics[width=0.30\textwidth]{Systemdesign/backend/klassebeskrivelser/Images/Backendproductcat.png}
    \caption{BackendProductCategory}
    \label{fig:BackendProductCategory}
\end{figure}
\end{center}
\label{BackendProductCategoryr_Beskrivelse}
 \bigskip 

