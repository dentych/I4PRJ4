\subsection{Design og Implementering}

Da \gls{SL} skulle designes blev det i første iteration klart at der ville opstå et væld af klasser i dette bibliotek. Det blev derfor besluttet at disse skulle være tilgængelige via en .dll fil som alle systemer fik adgang til og disse klasser blev efter grundige overvejelser indkapslet i de pakker som ses på figur \ref{fig:oversigtSL}. Pakkerne blev herefter udviklet efter nødvendighed igennem de forskellige iterationer, Hvilket satte SOLID designprincipperne på noget af en prøve.

\begin{figure}[!h]
    \centering
    \includegraphics[width=0.8\textwidth]{Systemdesign/SharedLib/Images/SharedLib_Package.png}
    \caption{Oversigt over alle pakker i SharedLib}
    \label{fig:oversigtSL}
\end{figure}

På figur \ref{fig:oversigtSL} ses det hvordan strukturen i SharedLib er opbygget.  \\

PUNKTFORM I FORHOLD TIL FIGUR!!!\\

FORTÆL HER NOGET OVERVEJELSER VED SOLID!!!\\

FORTÆL YDERLIGERE GENERELT OM SHAREDLIB!!!\\

\subsection{Models}
Moddellerne i \gls{SL} har fungeret som domæneklasser og har grundlæggende været brugt til at ensrette data på tværs af delsystemerne.

\subsubsection{Klassebeskrivelser}
Herunder ses de forskellige modelklasser i \gls{SL} og der vil til hver klasse være en uddybende forklaring om dennes funktionalitet og ansvar\\

\textbf{Catalogue}\\
Denne klasse indeholder en liste af produktkategorier, der hver især selv indeholder en liste af produkter, se afsnit \ref{PRODUCTCATEGORY} om ProductCategory for yderligere info om disse. Catalogue er det katalog af produkter systemet indeholder og ved boot af et delsystem vil dette som regel kalde en GetCatalogueDetails kommando, som vil give dem en instans af Catalogue med alle produkter i deres enkelte produktkategorier i systemet.

\begin{figure}[H]
    \centering
    \includegraphics[width=0.6\textwidth]{Systemdesign/SharedLib/Images/Klasser/Model/Catalogue.png}
    \caption{Catalogue}
    \label{fig:klasseModelCata}
\end{figure}

\textbf{Product}\\
Denne klasse er meget brugt igennem hele systemet, den blev tidligt i udviklingen fastlagt så systemerne kunne udvides omkring dette koncept. Product indeholder alle data som for systemet er relevant omkring et produkt.

\begin{figure}[H]
    \centering
    \includegraphics[width=0.4\textwidth]{Systemdesign/SharedLib/Images/Klasser/Model/Product.png}
    \caption{Product}
    \label{fig:klasseModelPrd}
\end{figure}


\textbf{ProductCategory}\label{PRODUCTCATEGORY}\\
Klassen ProductCategory kom ind i en senere iteration og gav muligheden for at placere produkter under samme kategori, f.eks. æble og pære under en kategori der hed "frugt".
ProductCategory har derfor som sagt et navn og derudover også en liste af produkter. 

\begin{figure}[H]
    \centering
    \includegraphics[width=0.4\textwidth]{Systemdesign/SharedLib/Images/Klasser/Model/ProductCategory.png}
    \caption{ProductCategory}
    \label{fig:klasseModelPrdCtg}
\end{figure}


\textbf{PurchasedProduct}\\
Klassen PurchasedProduct blev oprettet da der skulle implementeres funktionalitet omkring salg. Dette nødvendigjorde at der skulle et tidsstempel på produktet der var solgt og at der derudover skulle et antal (hvis flere af samme produkt var købt), prisen på daværende tidspunkt, da den nuværende pris på produktet godt kunne være ændret senere i databasen og til sidst en total pris, som bliver gjort op af antallet og stykprisen. Der blev derudover tilføjet et event til PurchasedProduct da denne også bliver brugt i \gls{KA}'s shoppinglist. Denne skulle vide hvornår antallet i en instans ændres, så grænsefladen kan ændre værdien på skærmen.

\begin{figure}[H]
    \centering
    \includegraphics[width=0.6\textwidth]{Systemdesign/SharedLib/Images/Klasser/Model/PurchasedProduct.png}
    \caption{PurchasedProduct}
    \label{fig:klasseModelPurPrd}
\end{figure}


\textbf{Purchase}\\
Denne klasse sørger for at registrere køb fra \gls{KA}. Purchase klassen skal altså simulere et samlet køb af en eller flere varer og bliver derfor også brugt til at danne kvitteringer. Dette betyder altså at klassen har en liste af PurchasedProducts og et tidsstempel for købet.

\begin{figure}[H]
    \centering
    \includegraphics[width=0.4\textwidth]{Systemdesign/SharedLib/Images/Klasser/Model/Purchase.png}
    \caption{Purchase}
    \label{fig:klasseModelPurch}
\end{figure}





\subsection{Protocol}\label{PROTOKOL}
Protokollen til systemet blev tidligt besluttet at skulle fungere ved at sende strenge af XML over en socket forbindelse. Da det kom til den reelle udvikling af hvordan denne XML string skulle genereres blev der opsat nogle begreber til videre udarbejdelse:

\begin{itemize}
\item \textbf{Encode} Der skulle kunne gives et forudbestemt data objekt til en encode funktion der herefter skulle sørge for at generere en passende XML string. 
\item \textbf{Decode} Der skulle ligeledes kunne gives en string med XML som en decode funktion herefter kunne danne et passende data objekt udfra. 
\item \textbf{Commands} De kommandoer der skulle videregives informationer om skulle fungere som objekter der på bagrund af informationen i dem, kunne handles på andetsteds.
\item \textbf{Marshallers} Der skulle ligge noget logik "bagved" protocol objektet, som her skulle sørge for den reelle ændring fra det XML til data og omvendt. Det blev derfor nødvendigt at lave en Marshaller for hver Command som protocol objektet kunne kalde på.
\end{itemize}

For at "legemligøre" denne protocol blev der udarbejdet en protocol klasse. Denne vil på sigt kunne sætte sin egen marshaller attribut, og dermed vælge hvilket sprog der parses om til. Men da der i dette projekt som udgangspunkt kun bruges XML, sættes Protocol klassens marshaller attribut altid til XmlMarshal klassen.

Protocol klassen vil udover kommandoer være det eneste der bliver stiftet bekendtskab med når der skal parses. Udviklere vil på den måde ikke skulle rode med marshallers og interfaces længere nede i lagene og på den måde kan alt funktionalitet omkring systemets protokol forholdsvis simpelt tilgås fra et højere abstraktionsniveau.

\begin{figure}[H]
	\centering
	\includegraphics[width=0.6\textwidth]{Systemdesign/SharedLib/Images/Klasser/Protocol.png}
	\caption{Protocol klassen}
	\label{fig:klasseProtocol}
\end{figure}


\begin{figure}[H]
    \centering
    \includegraphics[width=1.0\textwidth]{Systemdesign/SharedLib/Images/Protokol/Protocol_Sek.png}
    \caption{Sekvensdiagram over protokollens encode funktion}
    \label{fig:protocolSek}
\end{figure}

På figur \ref{fig:protocolSek} er der illustreret et scenarie hvor der bliver kaldt Encode med en CreateProduct kommando på protocol objektet, dette kan f.eks. være \gls{AS} der vil oprette et produkt i databasen og derfor skal sende denne besked over socket forbindelsen.

\gls{AS} vil derfor lave en instans af den givne kommando klasse, en instans af protocol klassen. Herefter kaldes Encode fra protocol objektet med kommandoen og denne vil herefter kalde Encode med kommandoen på den korrekte marshaller, som til sidst reelt parser objektet om til en XML string og sender dette tilbage gennem XmlMarshal klassen og tilbage til protocol objektet. 

Som det ses på figur \ref{fig:protocolSek}, er der ikke mange kald der bliver lavet når en kommando bliver parset, Dog sker der en masse logik i de enkelte klasser. Da protocol klassen blev udarbejdet blev det besluttet at der skulle være mulighed for udvidelse af systemet så der kunne sendes med andre dataformater end XML over socket forbindelserne. derfor blev der oprette et interface til XmlMarshal klassen som denne implementere og dette er derfor det første protocol tager stilling til.

Derefter er der et hav af commands og marshallers, og for at følge Open-Closed princippet skulle det på sigt være muligt at kunne sende andre commandoer over systemet. Derfor tager XmlMarshal stilling til, på baggrund af den indkomne kommandos navn, hvilken Marshaller der er passende for denne kommando og kalder, hvis denne eksisterer, encode/decode på denne marshaller.
\subsection{Commands}

Command tekst her, wooow

\subsection{Marshallers}

Marshaller tekst her, wooow

\subsubsection{Klassebeskrivelser}
Da UML repræsentationen af de forskellige marshallers praktisk talt ligner hinanden er diagrammerne i denne klassebeskrivelse slået sammen til oversigter over de tre typer af marshallers hvortil der vil være en kort beskrivelse af de enkelte klasser.\\

\textbf{Generelle marshallers} håndterer de generelle kommandoer. på figur \ref{fig:overklasseMarsGen} ses en oversigt over de forskellige generelle marshal klasser der alle implementerer ICmdMarshal interfacet.

\begin{figure}[H]
	\centering
	\includegraphics[width=0.8\textwidth]{Systemdesign/SharedLib/Images/Marshallers/GeneralMarshallers2.png}
	\caption{Oversigt over generelle marshallers der nedarver fra ICmdMarshal klassen}
	\label{fig:overklasseMarsGen}
\end{figure}

På figur \ref{fig:overklasseMarsGen} ses først \textit{RegisterPurchaseMarshal}, denne håndterer selvfølgelig RegisterPurchase kommandoen. Dennes encode/decode funktion skal derfor kunne tage højde for et Purchase objekt med en liste af PurchasedProducts og få disse omsat i korrekt format og ligeledes tilbage til de korrekte objekter. 

GetCatalogue kommandoen har i sig selv ingen logik, for yderligere information se figur \ref{fig:klasseCMDGetC}, men dette betyder dog stadig at \textit{GetCatalogueMarshal} skal parse kommando navnet, da det er dette \gls{CS} handler på baggrund af. 

\textit{CatalogueDetailsMarshal} håndterer CatalogueDetails kommandoen og skal derfor kunne parse en liste af ProductCategory objekter der hver især indeholder en liste af Product objekter. Dette gav i første omgang en del problemer da systemet fik indført produktkategorier da det praktisk talt krævede en total revidering af denne klasse.\\


\textbf{Produkt marshallers} håndterer alle de produkt specifikke kommandoer. på figur \ref{fig:overklasseMarsP} ses en oversigt over de forskellige produkt marshal klasser der alle implementerer ICmdMarshal interfacet.

\begin{figure}[H]
	\centering
	\includegraphics[width=0.8\textwidth]{Systemdesign/SharedLib/Images/Marshallers/ProductMarshallers2.png}
	\caption{Oversigt over produkt specifikke marshallers der implementerer ICmdMarshal interfacet}
	\label{fig:overklasseMarsP}
\end{figure}

På figur \ref{fig:overklasseMarsP} ser den oplyste læser hurtigt at produkt marshallerne på samme måde som kommandoerne kan sættes i par. \textit{CreateProductMarshal} og \textit{ProductCreatedMarshal} håndterer henholdsvis CreateProduct og ProductCreated kommandoerne og skal derfor sørge for at parse Product attributter til og fra XML.

Derefter kommer \textit{DeleteProductMarshal} og \textit{ProductDeletedMarshal} der håndterer henholdsvis DeleteProduct og ProductDeleted kommandoerne, disse skal ligeledes sørge for at parse simple attributter.

Sidst kommer \textit{EditProductMarshal} og \textit{ProductEditedMarshal} der håndterer henholdsvis EditProduct og ProductEdited kommandoerne, disse skal på samme måde som de tidligere marshallers af samme type parse produkt specifikke attributter til og fra XML.\\



\textbf{Produktkategori marshallers} håndterer alle de produktkategori specifikke kommandoer. på figur \ref{fig:overklasseMarsPC} ses en oversigt over de forskellige produktkategori marshal klasser der alle implementerer ICmdMarshal interfacet.

\begin{figure}[H]
	\centering
	\includegraphics[width=0.8\textwidth]{Systemdesign/SharedLib/Images/Marshallers/ProductCategoryMarshallers2.png}
	\caption{Oversigt over produktkategori specifikke marshallers der nedarver fra ICmdMarshal klassen}
	\label{fig:overklasseMarsPC}
\end{figure}

På figur \ref{fig:overklasseMarsP} ses det meget lignende produkt marshallerne at produktkategori marshallerne på samme måde som kommandoerne kan sættes i par. \textit{CreateProductCategoryMarshal} og \textit{ProductCategoryCreatedMarshal} håndterer henholdsvis CreateProductCategory og ProductCategoryCreated kommandoerne og skal derfor sørge for at parse et ProductCategory objekt med en liste af Product objekter og deres attributter til og fra XML.

Derefter kommer \textit{DeleteProductCategoryMarshal} og \textit{ProductCategoryDeletedMarshal} der håndterer henholdsvis DeleteProductCategory og ProductCategoryDeleted kommandoerne, disse skal ligeledes sørge for at parse et ProductCategory objekt med en liste af Product objekter og deres attributter.

Sidst kommer \textit{EditProductCategoryMarshal} og \textit{ProductCategoryEditedMarshal} der håndterer henholdsvis EditProductCategory og ProductCategoryEdited kommandoerne, disse skal på samme måde som de tidligere marshallers af samme type parse et ProductCategory objekt med en liste af Product objekter og deres attributter til og fra XML.\\



