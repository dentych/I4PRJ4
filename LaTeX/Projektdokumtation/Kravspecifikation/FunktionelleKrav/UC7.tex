\subsection{Use case 6: Opret produktkategori}


\begin{table}[H]
\begin{tabularx}{\textwidth}{|l|X|}
\hline
\textbf{Navn}					& Opret produktkategori \\\hline
\textbf{Formål}					& At oprette en produktkategori til produkter \\\hline
\textbf{Initialisering}			& \gls{SB} \\\hline
\textbf{Aktører}				& \gls{SB} (Primær)\\\hline
\textbf{Reference}				& Ingen \\\hline
								
\textbf{Samtidige forekomster}	& $\infty$ \\\hline

\textbf{Prækondition}			& Der er oprettet forbindelse til \gls{CS} en produktkategori i systemet. \\\hline

\textbf{Postkondition}			& Der er oprettet en \textit{ny} produktkategori i systemet \\\hline

\textbf{Hovedscenarie}			& 1. \gls{SB} trykker på knappen til opret produktkategori \\		
								& 2. Et pop up vindue åbnes i \gls{AS}'s \gls{GUI}, hvori der indtastes informationer om produktkategorien.\\
								& 3. \gls{SB} trykker på knappen til gem\\
								& ~ [Ext 1: Produktkategori eksisterer allerede]\\
								& 3. Systemet sender kommando med information til \gls{CS}. \\
								& ~ [Ext 2: \gls{CS} melder fejl.\\
								& 4. \gls{AS} modtager opdatering om ny produktkategori. \\
								& 5. \gls{AS} opdaterer kategorilisten \\\hline

\textbf{Extensions}				
								& [Ext 1: Produktkategori eksisterer allerede] \\	
								& ~ 1. Beskedboks informerer \gls{SB} om at produktkategorien eksisterer allerede.\\
								& ~ 2. Der fortsættes fra punkt 2.\\
								 		
								& [Ext 2: \gls{CS} melder fejl] \\
								& ~ 1. Beskedboks informerer \gls{SB} om fejl. \\
								& ~ 2. Usecasen afsluttes \\\hline
\end{tabularx}
\caption{Use case 6: Opret produktkategori}
\label{tab:UCopk}
\end{table}