\subsection{Use case 2: Returner vare}


\begin{table}[H]
\begin{tabularx}{\textwidth}{|l|X|}
\hline
\textbf{Navn}					& Returner vare \\\hline
\textbf{Formål}					& Returnere en fejlkøbt vare \\\hline
\textbf{Initialisering}			& \gls{Eks} \\\hline
\textbf{Aktører}				& \gls{Eks} (Primær) \\
								& \gls{Kunde} (Sekundær) \\\hline
\textbf{Reference}				& Ingen \\\hline
								
\textbf{Samtidige forekomster}	& Én per \gls{KA} \\\hline
\textbf{Prækondition}			& \gls{KA} har forbindelse til \gls{CS} \\\hline
\textbf{Postkondition}			& \gls{Kunde} har returneret sin vare og fået tilsvarende penge udbetalt \\\hline
\textbf{Hovedscenarie}			
& 1. \gls{Eks} indtaster varen \\			
& 2. \gls{Eks} vælger vare og indtaster antal og vælger Retur \\
& ~ [Ext 1: \gls{Kunde} fortryder] \\
& 3. \gls{Eks} indtaster 0 + kontant \\
& 4. \gls{Eks} giver \gls{Kunde} angivet beløbet tilbage \\\hline

\textbf{Extensions}				
& [Ext 1: \gls{Kunde} fortryder] \\
& 1. \gls{Eks}en vælger Annuller \\\hline
\end{tabularx}
\caption{Use case 2: Returner vare}
\label{tab:UCrtv}
\end{table}