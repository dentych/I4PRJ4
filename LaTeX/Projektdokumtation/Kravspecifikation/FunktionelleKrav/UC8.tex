\subsection{Use case 7: Rediger \gls{PDK}}


\begin{table}[H]
\begin{tabularx}{\textwidth}{|l|X|}
\hline
\textbf{Navn}					& Rediger \gls{PDK} \\\hline
\textbf{Formål}					& At redigere en instans af \gls{PDK} \\\hline
\textbf{Initialisering}			& \gls{SB} \\\hline
\textbf{Aktører}				& \gls{SB} (Primær)\\\hline
\textbf{Reference}				& Ingen \\\hline
								
\textbf{Samtidige forekomster}	& $\infty$ \\\hline

\textbf{Prækondition}			& Forbindelse til \gls{CS} er oprettet \\\hline

\textbf{Postkondition}			& Instansen af \gls{PDK} er redigeret i systemet \\\hline

\textbf{Hovedscenarie}			& 1. \gls{SB} trykker på den produktkategori der ønskes at redigere i listen af produktkategorier. \\		
								& 2. \gls{SB}en trykker på knappen til rediger produkt i hovedmenuen.\\
								& 3.  \gls{SB}en ændrer de ting der ønskes at redigeres, og trykker på knappen til gem. \\
								& 4. \gls{AS} modtager opdatering om redigeret produktkategori. \\
								& ~ [Ext 1: \gls{CS} melder fejl.\\
								& 5. \gls{AS} opdaterer listen af produktkategorier \\\hline

\textbf{Extensions}							
								& [Ext 1: \gls{CS} melder fejl]\\
								& ~ 1. Beskedboks informerer \gls{SB} om fejl.\\
								& ~ 2. Usecasen afsluttes.\\\hline
\end{tabularx}
\caption{Use case 7: Rediger \gls{PDK}}
\label{tab:UCrpk}
\end{table}