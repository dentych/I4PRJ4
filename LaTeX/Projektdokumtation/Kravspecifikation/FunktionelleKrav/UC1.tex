\subsection{Use case 1: Gennemfør salg}


\begin{table}[H]
\begin{tabularx}{\textwidth}{|l|X|}
\hline
\textbf{Navn}					& Gennemfør salg \\\hline
\textbf{Formål}					& Gennemføre et salg af et vilkårligt antal varer til \gls{Kunde} \\\hline
\textbf{Initialisering}			& \gls{Eks} \\\hline
\textbf{Aktører}				& \gls{Eks} (Primær), \\& \gls{Kunde} (Sekundær) \\\hline
\textbf{Reference}				& Ingen \\\hline
								
\textbf{Samtidige forekomster}	& Én per instans af \gls{KA} \\\hline
\textbf{Prækondition}			& \gls{KA} har forbindelse til \gls{CS} \\\hline
\textbf{Postkondition}			& \gls{Kunde} har købt sine varer og salget er gemt i \gls{DB} \\\hline
\textbf{Hovedscenarie}			
& 1. For hver vare \gls{Kunde} vil købe:\\		
& ~~ a. \gls{Eks} indtaster vare \\
& ~~ b. Ekspedient vælger vare og indtaster antal og vælger Antal \\
& ~ [Ext 1: Forkert vare indtastet] \\
& ~ [Ext 2: Forkert antal indtastet] \\
& ~ [Ext 3: Handlen afbrydes] \\
& 2. Kunde giver Ekspedient penge \\
& 3. Ekspedient indtaster beløbet, som Kunde har givet \\
& 4. Ekspedient trykker på knappen til valgte betalingsform \\
& 5. Kunden får eventuelt restbeløb tilbage \\
\hline

\textbf{Extensions}				
& [Ext 1: Forkert vare indtastet] \\
& ~ 1. Ekspedient vælger varen på liste over vare \\
& ~ 2. Ekspedient trykker på delete knappen \\
& [Ext 2: Forkert antal indtastet] \\
& ~ 1. Ekspedient vælger varen på liste over vare \\
& ~ 2. Ekspedient indtaster det korrekte antal \\
& ~ 3. Ekspedient trykker på “Antal” for at bekræfte \\
& [Ext 3: Handlen afbrydes] \\
& ~ 1. Ekspedient trykker på annuller knappen \\
\hline

\end{tabularx}
\caption{Use case 1: Gennemfør salg}
\label{tab:UCgfs}
\end{table}