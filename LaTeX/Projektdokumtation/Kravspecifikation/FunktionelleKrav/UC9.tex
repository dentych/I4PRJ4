\subsection{Use case 8: Slet produktkategori}

\begin{table}[H]
\begin{tabularx}{\textwidth}{|l|X|}
\hline
\textbf{Navn}					& Slet produktkategori \\\hline
\textbf{Formål}					& At slette en produktkategori \\\hline
\textbf{Initialisering}			& Superbruger \\\hline
\textbf{Aktører}				& Superbruger (Primær)\\\hline
\textbf{Reference}				& Ingen \\\hline
								
\textbf{Samtidige forekomster}	& $\infty$ \\\hline
\textbf{Prækondition}			& Forbindelse til serveren er oprettet \\\hline
\textbf{Postkondition}			& Produktkategorien er slettet i systemet \\\hline

\textbf{Hovedscenarie}			& 1. Superbruger trykker på den kategori der ønskes at slette i listen af kategorier. \\		
								& 2. Superbruger trykker på knappen til slet kategori i hovedmenuen og der vises et vindue.\\
								& 4. Superbruger vælger hvilken kategori han ønsker at flytte produkterne til og trykker på knappen til at flytte produkterne.\\
								& ~ [Ext 1: Der er ingen produkter i kategorien.]\\
								& ~ [Ext 2: Superbruger annullerer sletningen]\\
								& 5. Superbrugeren trykker på slet kategori, efter produkterne er flyttet.\\
								& 5. Vinduet lukker og kategorien bliver slettet fra listen over kategorier.\\
								& ~ [Ext 3: Central server melder fejl]\\
								& 8. Administrationssystemet opdaterer kategorilisten \\\hline

\textbf{Extensions}							
  								& [Ext 1: Der er ingen produkter i den valgte kategori.] \\
  								& ~ 1. Knappen til Flyt er grå, og kan ikke aktiveres.\\
  								& [Ext 2: Brugeren annullerer sletningen]\\
  								& ~ 2. Usecasen afsluttes \\
								
								& [Ext 3: Central server melder fejl]\\
								& ~ 1. Beskedboks informerer brugeren om fejl.\\
								& ~ 2. Usecasen afsluttes.\\\hline
\end{tabularx}
\caption{Use case 8: Slet produktkategori}
\label{tab:UCspk}
\end{table}