\subsection{Use case 5: Slet produkt}


\begin{table}[H]
\begin{tabularx}{\textwidth}{|l|X|}
\hline
\textbf{Navn}					& Slet produkt \\\hline
\textbf{Formål}					& At slette et produkt \\\hline
\textbf{Initialisering}			& \gls{SB} \\\hline
\textbf{Aktører}				& \gls{SB} \\\hline
\textbf{Reference}				& Ingen. \\\hline								
\textbf{Samtidige forekomster}	& $\infty$ \\\hline
\textbf{Prækondition}			& Der skal være et produkt at slette. \\\hline
\textbf{Postkondition}			& Det valgte produkt skal være slettet. \\
\hline
\textbf{Hovedscenarie}			& 1. \gls{SB} vælger det produkt der ønskes slettet, og trykker på knappen til at slette produkt.\\												
								& 2. Et dialogvindue, der spørger om produktet ønskes slettet, vises på skærmen.\\
								& 3. \gls{SB} trykker på knappen til at godkende sletningen og dialogvinduet forsvinder.\\
								& ~ [Ext 1: \gls{SB} vælger at annullere sletningen.]\\
								& 4. Produktet slettets umiddelbart herefter fra listen over produkt.\\
								& ~ [Ext 2: \gls{CS} melder fejl]\\
\hline
\textbf{Extensions}				& [Ext 1: \gls{SB} vælger at annullere sletningen.]\\
								& ~ 1. \gls{SB} trykker på knappen til at annullere sletningen.\\
								& ~ 2. Dialogvinduet forsvinder og intet yderligere foretages.\\
								& [Ext 2: \gls{CS} melder fejl] \\
								& ~ 1. Beskedboks informerer \gls{SB} om fejl. \\
								& ~ 2. Usecasen afsluttes \\\hline
\hline
\end{tabularx}
\caption{Use case 5: Slet produkt}
\label{tab:UCsp}
\end{table}