\subsection{Use case 3: Opret produkt} \label{prodkast}
At oprette et produkt til systemet.



\begin{table}[H]
\begin{tabularx}{\textwidth}{|l|X|}
\hline
\textbf{Navn}					& Opret produkt \\\hline
\textbf{Formål}					& \gls{SB} opretter produkt. \\\hline
\textbf{Initialisering}			& \gls{SB} \\\hline
\textbf{Aktører}				& \gls{SB} (Primær)\\\hline
\textbf{Reference}				& Ingen \\\hline
								
\textbf{Samtidige forekomster}	& $\infty$ \\\hline
\textbf{Prækondition}			& Der er oprettet forbindelse til \gls{CS}.
\\\hline
\textbf{Postkondition}			& Der er oprettet et nyt, tidligere ubenyttet produkt i systemet.
\\\hline
\textbf{Hovedscenarie}			& 1. \gls{SB} trykker på opret produkt på i \gls{AS}.\\												& 2. Et pop up vindue åbnes i \gls{GUI}, hvori der indtastes informationer om produktet.\\
								& 3. \gls{SB} trykker på gem.\\
								& ~ [Ext 1: Produkt eksisterer allerede] \\
								& 2. Systemet sender kommando til \gls{CS}.\\
								& ~ [Ext 2: \gls{CS} melder fejl.\\
								& 3. \gls{AS} modtager opdatering om nyt produkt. \\
								& 4. \gls{AS} opdaterer produktlisten \\\hline

\textbf{Extensions}						
								& [Ext 1: produktet eksisterer allerede] \\
								& ~ 1. Beskedboks informerer \gls{SB} om at produktet allerede eksisterer.\\
								& ~ 2. Der fortsættes fra punkt 1.\\
									
								& [Ext 2: \gls{CS} melder fejl] \\
								& ~ 1. Beskedboks informerer \gls{SB} om fejl. \\
								& ~ 2. Usecasen afsluttes \\\hline
\end{tabularx}
\caption{Use case 3: Usecase for opret produkt}
\label{tab:UCop}
\end{table}
