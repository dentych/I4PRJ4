\subsection{Use case 4: Rediger produkt}


\begin{table}[H]
\begin{tabularx}{\textwidth}{|l|X|}
\hline
\textbf{Navn}					& Rediger produkt \\\hline
\textbf{Formål}					& At produkt bliver redigeret \\\hline
\textbf{Initialisering}			& \gls{SB} \\\hline
\textbf{Aktører}				& \gls{SB} \\\hline
\textbf{Reference}				& Ingen \\\hline
								
\textbf{Samtidige forekomster}	& $\infty$ \\
\hline
\textbf{Prækondition}			& Der er oprettet forbindelse til \gls{CS} \\
\hline
\textbf{Postkondition}			& Det valgte produkt er opdateret med ny data \\
\hline
\textbf{Hovedscenarie}			& 1. \gls{SB} vælger det produkt der ønskes redigeret, og trykker på knappen til at redigere produkt.\\
								& 2. Et nyt vindue åbner, med det valgte produkts nuværende informationer.\\
								& 3. \gls{SB} redigerer det data der ønskes ændret og trykker på knappen til gem.\\
								& ~ [Ext 1: \gls{SB} vælger at annullere ændringen]\\
								& ~ [Ext 2: \gls{CS} melder fejl]\\
								& 4. Vinduet lukker og det valgte produkt opdateres med det nye data.\\
\hline
\textbf{Extensions}				& [Ext 1: \gls{SB} vælger at annullere ændringen.] \\
								& ~ 1. \gls{SB} trykker på knappen til at annullere.\\
								& ~ 2. Vinduet til at redigere data lukker, hovedmenuen vises og det valgte produkt forbliver uændret.\\
								& [Ext 2: \gls{CS} melder fejl] \\
								& ~ 1. Beskedboks informerer \gls{SB} om fejl. \\
								& ~ 2. Usecasen afsluttes \\\hline
								
\end{tabularx}
\caption{Use case 4: Rediger produkt}
\label{tab:UCrp}
\end{table}